\chapter*{Abstract}
\addcontentsline{toc}{chapter}{Abstract}

The exponential growth of cyber threats combined with the proliferation of encrypted communications and distributed computing environments has created unprecedented challenges for network security. This dissertation presents a comprehensive investigation of advanced machine learning approaches for network intrusion detection systems, addressing fundamental limitations of traditional signature-based and rule-based detection methods.

This research makes several key contributions spanning theoretical foundations, algorithmic innovations, and empirical validation across diverse security domains. The work introduces novel applications of continuous-time neural architectures, optimal transport theory, privacy-preserving federated learning, and hybrid spatial-temporal modeling to address critical challenges in modern cybersecurity.

The first major contribution develops Temporal Adaptive Batch Normalization Neural Ordinary Differential Equations (TA-BN-ODE) integrated with Deep Spatio-Temporal Point Processes for real-time adaptive intrusion detection. This framework models both continuous system dynamics and discrete attack occurrences across eight orders of magnitude in time scales, achieving 97.3\% accuracy with 60--90\% parameter reduction through continuous-depth adaptation. The approach processes 12.3 million events per second with sub-100 millisecond latency, demonstrating structured variational Bayesian inference that provides calibrated confidence intervals with 91.7\% coverage probability. Integration with large language models enables zero-shot detection of novel attack patterns with 87.6\% F1-score, while spiking neural network conversion reduces energy consumption by 73\% for edge deployment.

The second major contribution introduces Privacy-Preserving Federated Optimal Transport for multi-cloud intrusion detection (PPFOT-IDS), representing the first application of optimal transport theory to network security under differential privacy constraints. The framework achieves $(\epsilon = 0.85, \delta = 10^{-5})$-differential privacy while maintaining 94.2\% detection accuracy across heterogeneous cloud environments. Adaptive entropy regularization with Sinkhorn divergence reduces computational complexity from cubic to quadratic scaling, enabling real-time adaptation with 15--23$\times$ speedup. Byzantine-robust aggregation tolerates up to 40\% malicious participants while maintaining 87.1\% accuracy through transport-plan-based anomaly detection.

The third major contribution develops hybrid spatial-temporal deep learning architectures for privacy-preserving encrypted traffic analysis without requiring decryption. Combining convolutional neural networks for spatial feature extraction with long short-term memory networks for temporal modeling, the approach achieves 97--99.9\% detection accuracy across threat categories including malware, botnets, advanced persistent threats, and distributed denial of service attacks. Transformer-based architectures with self-attention mechanisms capture long-range dependencies, while federated learning frameworks enable privacy-preserving distributed detection across organizational boundaries.

Additional contributions include federated learning approaches utilizing graph temporal dynamics, knowledge distillation for model compression, and Byzantine-robust aggregation mechanisms. Graph-based methods leverage network topology through heterogeneous graph pooling and attention mechanisms to capture attack propagation patterns across network structures.

Comprehensive experimental validation is conducted on the Integrated Cloud Security 3Datasets (ICS3D) comprising 18.9 million security records across container orchestration, IoT/IIoT networks, and enterprise security operations, alongside standard benchmarks including CIC-IDS2018, UNSW-NB15, and CIC-IoT-2023. The integrated approaches demonstrate consistent performance improvements of 15--21\% over baseline methods while maintaining privacy guarantees and computational efficiency suitable for operational deployment.

Cross-domain validation on speech event detection and healthcare monitoring confirms broad applicability of the continuous-discrete hybrid modeling paradigm beyond network security contexts. Theoretical analysis provides convergence guarantees for online learning under concept drift, PAC-Bayesian generalization bounds, and differential privacy preservation for collaborative threat intelligence sharing.

This dissertation establishes that modern machine learning approaches, when carefully designed with domain-specific constraints and theoretical rigor, enable transformative improvements in network security while addressing critical operational requirements including privacy preservation, computational efficiency, adversarial robustness, and adaptation to evolving threats. The integrated framework advances the state of the art in intrusion detection systems while providing practical solutions deployable in contemporary distributed computing environments.

\vspace{1cm}

\noindent\textbf{Keywords:} Network intrusion detection, deep learning, neural ordinary differential equations, optimal transport, temporal point processes, differential privacy, federated learning, encrypted traffic analysis, Bayesian inference, graph neural networks, cybersecurity, Byzantine robustness, continuous-time models, edge computing, zero-shot learning.

\cleardoublepage
