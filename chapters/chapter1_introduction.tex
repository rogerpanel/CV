\chapter{Introduction}
\label{ch:introduction}

\section{Motivation and Background}

The landscape of cybersecurity has undergone profound transformation over the past decade, driven by the convergence of several technological and societal trends. The exponential growth of interconnected devices through the Internet of Things, the migration of enterprise computing to multi-cloud environments, the widespread adoption of encryption protocols for privacy protection, the emergence of post-quantum cryptographic standards, the proliferation of microservices architectures, and the increasing sophistication of cyber attacks have collectively created unprecedented challenges for network security practitioners and researchers.

Network intrusion detection systems represent a critical component of cybersecurity infrastructure, serving as the first line of defense against malicious activities in digital networks. Traditional approaches to intrusion detection, primarily relying on signature-based pattern matching and rule-based anomaly detection, face fundamental limitations when confronted with modern threat landscapes. These conventional methods struggle to detect zero-day exploits, cannot analyze encrypted traffic without compromising privacy, fail to adapt to evolving attack strategies, exhibit poor generalization across heterogeneous network environments, and lack the architectural sophistication to capture multi-granularity attack patterns spanning from individual service instances to distributed system behaviors.

The emergence of machine learning and deep learning technologies has opened new avenues for addressing these limitations. However, direct application of standard machine learning techniques to network security introduces its own set of challenges. Network traffic exhibits complex temporal dependencies spanning multiple time scales, from microsecond-level timing attacks to month-long advanced persistent threat campaigns. Security datasets suffer from severe class imbalance, with malicious traffic typically representing less than five percent of total network flows. Privacy regulations and competitive concerns prevent centralized data aggregation necessary for training comprehensive models. Resource constraints on edge devices and IoT gateways limit the deployment of computationally expensive deep learning architectures. Adversarial actors actively manipulate detection systems through poisoning attacks and adversarial examples. The advent of quantum computing threatens current cryptographic foundations, necessitating new approaches that account for post-quantum security paradigms. Modern cloud-native applications deployed as microservices create multi-granularity security challenges where threats manifest simultaneously at service, trace, and pod levels.

\section{Research Problems and Challenges}

This dissertation addresses nine fundamental research problems at the intersection of machine learning and network security, spanning continuous-time modeling, privacy-preserving learning, encrypted traffic analysis, federated intelligence, graph-based detection, post-quantum security, large language model integration, and multi-granularity microservices protection:

\textbf{Problem 1: Temporal Modeling of Network Attacks.} Network attacks exhibit complex temporal dynamics that cannot be adequately captured by discrete-time architectures operating at fixed sampling intervals. Consider an advanced persistent threat conducting reconnaissance through carefully randomized port probes spanning intervals from 10 seconds to 60 minutes, specifically designed to evade fixed-window detectors. Traditional systems sampling every minute miss 73\% of such probes. The fundamental challenge lies in developing models that jointly capture continuous system state evolution and discrete event occurrences across eight orders of magnitude in time scales, while maintaining computational tractability and theoretical stability guarantees.

\textbf{Problem 2: Multi-Cloud Domain Adaptation Under Privacy Constraints.} Multi-cloud deployments create critical domain adaptation challenges where intrusion detection systems trained on one cloud provider must generalize to heterogeneous environments without sharing sensitive security data. Microsoft's 2024 State of Multicloud Security Risk Report documents that the average multi-cloud deployment contains 351 exploitable attack paths, with security models achieving only 54 to 67 percent accuracy when transferred across cloud boundaries without adaptation. The challenge involves developing principled distribution alignment methods that maintain strong privacy guarantees while enabling effective threat detection across clouds, incorporating optimal transport theory to measure and minimize distributional discrepancies under differential privacy constraints.

\textbf{Problem 3: Encrypted Traffic Analysis Without Decryption.} The proliferation of encrypted network communications, with over 85.9\% of cyberattacks utilizing encrypted channels, has rendered traditional deep packet inspection ineffective. Decrypting traffic introduces privacy violations, legal complications under regulations such as GDPR, and computational overhead prohibitive at network scale. The research problem requires developing approaches that learn discriminative patterns from observable traffic metadata without accessing encrypted payload contents, while achieving detection accuracy comparable to systems with full payload visibility.

\textbf{Problem 4: Privacy-Preserving Collaborative Threat Intelligence.} Effective threat detection benefits from learning diverse attack patterns across organizations, yet security data cannot be centrally aggregated due to privacy regulations and competitive concerns. Federated learning provides a framework for collaborative model training, but faces challenges including severe distribution heterogeneity across participants, Byzantine attacks from malicious nodes, and communication overhead incompatible with real-time detection requirements. The fundamental problem involves designing federated architectures with provable convergence guarantees under adversarial conditions while minimizing communication costs through efficient aggregation mechanisms.

\textbf{Problem 5: Zero-Shot Detection and Adaptation to Novel Threats.} The continuous evolution of cyber attacks through new exploit techniques, malware variants, and attack strategies renders purely supervised learning approaches inadequate. Security systems must detect previously unseen attack types without labeled training examples while adapting to concept drift from software updates and infrastructure changes. This necessitates developing models with few-shot learning capabilities and semantic understanding of attack principles rather than mere pattern matching, leveraging recent advances in large language models for zero-shot threat detection.

\textbf{Problem 6: Continuous-Time Temporal Graph Neural Networks for Dynamic Security Graphs.} Network security inherently involves graph-structured data representing service dependencies, communication patterns, and attack propagation paths. However, these security graphs exhibit continuous temporal evolution where edges appear and disappear dynamically, node features evolve continuously, and critical events occur at irregular intervals. Traditional graph neural networks operate on static snapshots, losing temporal information and introducing discretization artifacts. The research challenge involves developing continuous-time temporal graph neural networks that jointly model graph structure evolution and node dynamics through neural ordinary differential equations, enabling detection of sophisticated attacks that exploit temporal graph patterns such as lateral movement and privilege escalation across time-varying service dependencies.

\textbf{Problem 7: Large Language Model Federation for Zero-Day API Threat Detection.} The proliferation of RESTful APIs and microservices architectures has created vast attack surfaces where traditional signature-based detection fails against zero-day API vulnerabilities. Modern APIs exhibit complex behavioral patterns encoded in request sequences, parameter dependencies, and authentication flows that require semantic understanding beyond statistical anomaly detection. Furthermore, API security data across organizations cannot be centrally shared due to privacy concerns and competitive advantage considerations. The fundamental challenge lies in developing privacy-preserving federated learning frameworks that integrate large language models for zero-shot detection of novel API attacks while maintaining differential privacy guarantees, Byzantine robustness, and communication efficiency through parameter-efficient fine-tuning and prompt-based knowledge aggregation.

\textbf{Problem 8: Adversarially Robust Intrusion Detection for Post-Quantum Encrypted Traffic.} The NIST standardization of post-quantum cryptographic algorithms including ML-KEM (Kyber), ML-DSA (Dilithium), and SLH-DSA (SPHINCS+) in August 2024 marks a fundamental shift in network security. Post-quantum TLS connections exhibit distinct traffic patterns with significantly larger handshake sizes, different timing characteristics, and novel protocol structures that existing intrusion detection systems cannot effectively analyze. Moreover, quantum-enhanced adversaries can leverage Grover's algorithm for efficient adversarial example generation with quadratic speedup, fundamentally threatening classical machine learning defenses. The research problem requires developing hybrid classical-quantum machine learning architectures that jointly model classical and post-quantum traffic patterns while providing certified adversarial robustness against quantum-enhanced attacks through techniques including Lipschitz-constrained networks, quantum noise injection, and randomized smoothing with provable defense radii.

\textbf{Problem 9: Multi-Granularity Microservices Security Through Triple-Embedding Temporal Graphs.} Modern cloud-native applications deployed as microservices create multi-granularity security challenges where attacks manifest simultaneously across three complementary levels. At the service-level granularity, threats appear as aggregate behavioral anomalies in request rates, error distributions, and resource utilization. At the trace-level granularity, distributed attacks create suspicious request flow patterns violating expected service call graphs and introducing abnormal latencies. At the node-level granularity, compromised container pods exhibit resource exhaustion, malicious processes, and unauthorized network connections. Existing intrusion detection approaches operate at single granularities, missing coordinated attacks that span multiple levels. The fundamental challenge involves developing triple-embedding temporal graph neural networks that jointly learn hierarchical representations at service, trace, and node granularities through specialized encoders, integrate these multi-scale embeddings via heterogeneous temporal graph neural networks, and provide adaptive fusion mechanisms that dynamically weight granularity contributions based on attack characteristics.

\section{Research Objectives}

The overarching objective of this dissertation is to develop a comprehensive framework of advanced machine learning approaches that address the fundamental challenges identified above while meeting operational requirements for network intrusion detection systems. Specific research objectives include:

\begin{enumerate}[leftmargin=*]
\item To develop continuous-time neural architectures that model temporal dependencies in network security events across multiple time scales, integrating Neural Ordinary Differential Equations with temporal point processes for unified continuous-discrete modeling.

\item To establish theoretical foundations and practical algorithms for optimal transport-based domain adaptation in multi-cloud intrusion detection under differential privacy constraints, with Byzantine-robust aggregation mechanisms.

\item To design hybrid spatial-temporal deep learning architectures for encrypted traffic analysis that achieve high detection accuracy using only observable metadata features without requiring decryption.

\item To create federated learning frameworks for distributed intrusion detection that enable privacy-preserving collaborative threat intelligence sharing while maintaining convergence guarantees under heterogeneous data distributions and adversarial participants.

\item To integrate large language models with temporal reasoning capabilities for zero-shot detection of novel attack patterns through semantic understanding rather than supervised pattern matching.

\item To develop continuous-time temporal graph neural networks for dynamic security graphs that capture attack propagation patterns across time-varying service dependencies and network topologies.

\item To establish privacy-preserving federated learning frameworks integrating large language models for zero-day API threat detection with differential privacy guarantees, Byzantine robustness, and parameter-efficient fine-tuning mechanisms.

\item To design hybrid classical-quantum machine learning architectures for adversarially robust intrusion detection on post-quantum encrypted traffic with certified defense guarantees against quantum-enhanced adversaries.

\item To create triple-embedding temporal graph neural networks for multi-granularity microservices security that jointly model service-level, trace-level, and node-level attack patterns through hierarchical representations and adaptive cross-granularity fusion.

\item To provide comprehensive experimental validation across diverse security domains including container orchestration, IoT/IIoT networks, encrypted traffic, multi-cloud environments, post-quantum networks, microservices architectures, and enterprise security operations, demonstrating practical deployment viability.

\item To establish theoretical guarantees including convergence analysis, generalization bounds, privacy preservation, adversarial robustness, and certified defense radii for the developed approaches.
\end{enumerate}

\section{Key Contributions}

This dissertation makes several key contributions spanning theoretical foundations, algorithmic innovations, architectural designs, and empirical validation across nine complementary research directions:

\subsection{Theoretical Contributions}

\begin{itemize}[leftmargin=*]
\item Development of Temporal Adaptive Batch Normalization Neural ODEs with stability analysis through Lyapunov theory, resolving the fundamental incompatibility between batch normalization and continuous dynamics in neural networks.

\item Integration of Neural ODEs with Deep Spatio-Temporal Point Processes for unified modeling of continuous system state evolution and discrete attack event occurrences, establishing convergence guarantees for the combined dynamics.

\item Formulation of privacy-preserving optimal transport for multi-cloud intrusion detection with formal $(\epsilon,\delta)$-differential privacy guarantees and utility-preserving bounds characterizing the privacy-accuracy trade-off under Wasserstein distance minimization.

\item Development of Byzantine-robust federated aggregation with provable convergence analysis showing convergence to within $O(\sqrt{q/K})$ of optimal global model under fraction $q$ of Byzantine adversaries across $K$ clients.

\item Establishment of PAC-Bayesian generalization bounds for security-critical decision making with calibrated confidence intervals achieving 91.7\% coverage probability across diverse threat categories.

\item Theoretical analysis of online learning convergence under concept drift with adaptive learning rates achieving sublinear regret bounds even with distribution shift in network traffic patterns.

\item Formulation of continuous-time temporal graph neural networks through neural graph ordinary differential equations, establishing existence and uniqueness theorems for the graph dynamics with stability analysis under Lipschitz continuity assumptions.

\item Development of parameter-efficient federated learning theory for large language models through Low-Rank Adaptation (LoRA), proving communication complexity reduction from $O(d^2)$ to $O(dr)$ where $r \ll d$ while maintaining model expressiveness.

\item Establishment of certified adversarial robustness for quantum-classical hybrid models through randomized smoothing, proving certified defense radius $R = \frac{\sigma}{2}(\Phi^{-1}(p_A) - \Phi^{-1}(p_B))$ against quantum-enhanced adversaries with Grover speedup.

\item Formulation of multi-granularity representation learning theory for hierarchical security graphs, proving that service-level, trace-level, and node-level embeddings capture complementary attack signals with bounded redundancy measured through mutual information.
\end{itemize}

\subsection{Algorithmic and Architectural Contributions}

\begin{itemize}[leftmargin=*]
\item Security-specific TA-BN-ODE architectures achieving 97.3\% accuracy with 60 to 90 percent parameter reduction through continuous-depth adaptation that allocates computational resources proportional to input complexity.

\item Transformer-enhanced marked temporal point processes with logarithmic barrier optimization reducing computational complexity from $O(n^3)$ to $O(n^2)$ in sequence length while capturing multi-scale temporal patterns spanning microseconds to months.

\item Privacy-Preserving Federated Optimal Transport (PPFOT-IDS) framework with adaptive entropy regularization achieving 15 to 23 times computational speedup through Sinkhorn divergence while maintaining strong privacy guarantees with $\epsilon = 0.5$, $\delta = 10^{-5}$.

\item Hybrid spatial-temporal architectures combining convolutional neural networks for spatial feature extraction with long short-term memory networks for temporal modeling, achieving 97 to 99.9 percent detection accuracy on encrypted traffic without decryption.

\item Integration of large language models with carefully designed prompt engineering achieving 87.6\% F1-score on zero-shot detection of novel attack patterns absent from training data.

\item Spiking neural network conversion for edge deployment reducing energy consumption by 73\% while maintaining 98 percent or higher accuracy on resource-constrained IoT devices.

\item Continuous-Time Temporal Graph Neural Network (CT-TGNN) architecture integrating graph structure evolution with node dynamics through neural graph ODEs, achieving 96.4\% accuracy on microservices attack detection with temporal modeling of service dependency changes.

\item Privacy-Preserving Federated LLM for API Security (FedLLM-API) framework integrating DistilBERT with LoRA adapters and prompt-based aggregation, achieving 97.2\% zero-day API threat detection accuracy with 99.97\% communication reduction and $\epsilon = 0.5$ differential privacy.

\item Post-Quantum Intrusion Detection System (PQ-IDPS) architecture combining CNN-LSTM classical pathway with 12-qubit Variational Quantum Classifier and adaptive fusion, achieving 95.3\% accuracy on hybrid post-quantum traffic with 91.7\% robustness under quantum adversarial attacks and certified defense radius $R = 0.42$.

\item Triple-Embedding Temporal Graph Neural Network (TripleE-TGNN) architecture with service-level GAT encoders, trace-level time-aware path encoding, node-level temporal graph convolution, and heterogeneous graph integration with adaptive granularity fusion, achieving 96.8\% accuracy on 41-service microservices with detection of coordinated multi-granularity attacks.
\end{itemize}

\subsection{Empirical Contributions}

\begin{itemize}[leftmargin=*]
\item Comprehensive experimental validation on the Integrated Cloud Security Datasets (ICS3D) comprising 18.9 million security records across container orchestration (697,289 Kubernetes flows), IoT/IIoT networks (4 million records from seven-layer testbed), and enterprise security operations (1 million alerts from 6,100 organizations with MITRE ATT\&CK annotations).

\item Evaluation on standard benchmarks including CIC-IDS2018, UNSW-NB15, NSL-KDD, and CIC-IoT-2023 enabling direct comparison with published baselines, demonstrating 15 to 21 percent accuracy improvements over state-of-the-art approaches.

\item Cross-domain validation on speech event detection (LibriSpeech achieving 94.2\% F1-score) and healthcare monitoring (MIMIC-III, eICU) confirming broad applicability of continuous-discrete hybrid modeling beyond network security.

\item Demonstration of real-time operational viability through processing 12.3 million events per second with sub-100 millisecond detection latency suitable for production deployment.

\item Extensive evaluation on microservices datasets including Train-Ticket (41 services, 2.4M requests), Sock-Shop (14 services), Online-Boutique (11 services), and DeathStarBench suite, demonstrating effectiveness on realistic cloud-native architectures.

\item Comprehensive assessment on federated API security across AWS API Gateway, Azure API Management, Google Cloud API, GraphQL APIs, and microservices architectures, validating cross-organizational threat intelligence sharing with 97.2\% accuracy.

\item Evaluation on post-quantum cryptographic datasets including CESNET-TLS-22 (2.1M hybrid classical-PQC connections), QUIC-PQC (847K pure Kyber connections), and IoT-PQC (1.6M Dilithium-signed connections), demonstrating 95.3\% detection on hybrid traffic.

\item Analysis of attack-specific performance across 12 threat categories including DDoS (98.2\% accuracy), privilege escalation (97.4\%), API abuse (95.8\%), container breakout (94.7\%), and lateral movement (96.1\%), quantifying granularity-specific detection strengths.

\item Ablation studies quantifying the contribution of individual components including temporal adaptive normalization, multi-scale point process modeling, optimal transport alignment, Byzantine-robust aggregation, zero-shot language model integration, continuous-time graph dynamics, parameter-efficient federated learning, quantum-classical fusion, and multi-granularity embedding.

\item Robustness evaluation under concept drift scenarios including service topology evolution (94.7\% maintained accuracy), API schema changes, post-quantum protocol transitions, and microservices scaling events, demonstrating adaptability to production environment dynamics.
\end{itemize}

\section{Research Methodology}

The research methodology employed throughout this dissertation follows a systematic approach integrating theoretical analysis, algorithm design, implementation, and empirical validation:

\begin{enumerate}[leftmargin=*]
\item \textbf{Problem Formulation:} Each research problem is rigorously formulated using mathematical notation, establishing clear objectives, constraints, and evaluation criteria. Formal definitions are provided for temporal graphs, privacy-preserving metrics, post-quantum security models, and multi-granularity representations.

\item \textbf{Theoretical Development:} Mathematical frameworks are developed drawing from continuous-time dynamical systems, optimal transport theory, differential privacy, Bayesian inference, quantum computing, graph theory, and statistical learning theory to provide principled foundations for algorithmic design. Convergence proofs, privacy analyses, and robustness guarantees are established through rigorous mathematical derivations.

\item \textbf{Algorithm Design:} Novel algorithms are designed based on theoretical foundations, incorporating domain-specific requirements from network security including real-time constraints, privacy preservation, adversarial robustness, computational efficiency, multi-granularity modeling, and quantum resistance. Pseudocode specifications are provided for computational reproducibility.

\item \textbf{Implementation:} Algorithms are implemented using modern deep learning frameworks including PyTorch, PyTorch Geometric, TensorFlow, and PennyLane for quantum simulation, with careful attention to numerical stability, computational efficiency, and reproducibility. Complete codebases with professional documentation are provided for all major contributions.

\item \textbf{Experimental Validation:} Comprehensive experiments are conducted on diverse security datasets spanning multiple domains, with rigorous evaluation using standard metrics including accuracy, precision, recall, F1-score, false positive rate, area under ROC curve, detection latency, and computational overhead. Statistical significance testing validates claimed improvements.

\item \textbf{Comparative Analysis:} Results are compared against established baselines and state-of-the-art methods including TGAT, DySAT, CONTINUUM, standard GNN architectures, classical federated learning, traditional encrypted traffic analysis, and existing microservices security approaches through standardized benchmarks.

\item \textbf{Ablation Studies:} Systematic ablation studies quantify the contribution of individual components across all nine research directions, providing insights into which design choices drive performance improvements and validating architectural decisions.

\item \textbf{Cross-Domain Evaluation:} Approaches are evaluated on domains beyond network security to assess generalizability and identify fundamental versus domain-specific contributions, including speech processing, healthcare monitoring, and general temporal event detection.

\item \textbf{Robustness Analysis:} Extensive robustness evaluations assess performance under adversarial attacks, concept drift, Byzantine participants, quantum-enhanced threats, and operational deployment conditions to validate practical viability.
\end{enumerate}

\section{Thesis Organization}

The remainder of this dissertation is organized as follows:

\textbf{Chapter 2} establishes the mathematical foundations and problem formulations for advanced intrusion detection systems. The chapter rigorously defines nine fundamental research problems through formal mathematical notation, including temporal modeling of network attacks, multi-cloud domain adaptation under privacy constraints, encrypted traffic analysis, privacy-preserving federated learning, uncertainty quantification, continuous-time temporal graphs, large language model federation for API security, post-quantum adversarial robustness, and multi-granularity microservices security. Each problem includes mathematical definitions with detailed notation, theoretical analysis with proofs where applicable, and justification of the chosen approaches. Novel mathematical formulations are introduced for continuous-time graph dynamics, federated LLM optimization, quantum-classical hybrid learning, and hierarchical multi-granularity representations.

\textbf{Chapter 3} presents a comprehensive literature review and theoretical foundations, surveying prior work in intrusion detection systems, machine learning for cybersecurity, neural ordinary differential equations, optimal transport theory, differential privacy, federated learning, graph neural networks, temporal modeling, large language models for security, post-quantum cryptography, quantum machine learning, microservices security, and related areas. The chapter synthesizes recent advances from 2024 to 2025 research including temporal graph learning, multi-granularity graph embeddings, heterogeneous graph neural networks, federated LLM training, post-quantum cryptanalysis, and hybrid quantum-classical architectures. Research gaps are identified that motivate the contributions of this dissertation.

\textbf{Chapter 4} develops Temporal Adaptive Neural ODEs with Deep Spatio-Temporal Point Processes for real-time network intrusion detection. The chapter introduces continuous-depth neural architectures with stability guarantees through Lyapunov analysis, multi-scale temporal point process modeling with intensity functions capturing attack patterns across time scales, and structured variational Bayesian inference for uncertainty quantification. Integration with large language models for zero-shot detection and spiking neural network conversion for edge deployment are also presented. Detailed algorithmic pseudocode and architectural diagrams illustrate the approach.

\textbf{Chapter 5} introduces Differentially Private Optimal Transport for multi-cloud intrusion detection, establishing the first application of optimal transport theory to network security under privacy constraints. The chapter develops the PPFOT-IDS framework with adaptive Sinkhorn optimization achieving 15 to 23 times speedup, Byzantine-robust aggregation with convergence guarantees, and formal privacy-utility trade-off analysis. The Wasserstein barycenter formulation enables distribution alignment across heterogeneous cloud environments while maintaining $(\epsilon, \delta)$-differential privacy. Mathematical proofs establish convergence properties and privacy preservation.

\textbf{Chapter 6} presents hybrid deep learning architectures for privacy-preserving encrypted traffic analysis without requiring decryption. The chapter develops spatial-temporal models combining convolutional and recurrent networks, transformer architectures with self-attention mechanisms, and federated learning frameworks for distributed encrypted traffic detection. A major new contribution integrates post-quantum cryptographic traffic analysis through the PQ-IDPS framework, combining classical CNN-LSTM pathways for traditional encrypted traffic with variational quantum classifiers for post-quantum protocol patterns. The hybrid classical-quantum architecture achieves 95.3\% accuracy on hybrid TLS 1.3 traffic containing both classical ECDHE and post-quantum Kyber key exchanges, with certified robustness radius $R = 0.42$ against quantum-enhanced adversaries through randomized smoothing. Detailed mathematical formulations describe quantum feature encoding, parameterized quantum circuits, Lipschitz-constrained networks for adversarial defense, and quantum noise injection mechanisms.

\textbf{Chapter 7} develops federated learning approaches for distributed intrusion detection, including graph temporal dynamics modeling, knowledge distillation for model compression, and Byzantine-robust aggregation mechanisms. The chapter establishes convergence guarantees under heterogeneous data distributions and adversarial participants. A substantial extension introduces the FedLLM-API framework for privacy-preserving large language model federation targeting zero-day API threat detection. The architecture integrates DistilBERT backbones with Low-Rank Adaptation for parameter-efficient fine-tuning, differential privacy encoders achieving $\epsilon = 0.5$ privacy, and attention-weighted Byzantine-robust aggregation. Novel prompt-based knowledge aggregation reduces communication overhead by 99.97\% compared to full model sharing. Mathematical analysis establishes convergence to within $O(\sqrt{q/K})$ of optimal under Byzantine ratio $q$ across $K$ organizations. Experimental results on AWS, Azure, and Google Cloud API datasets demonstrate 97.2\% zero-day detection accuracy with 68\% communication reduction through LoRA.

\textbf{Chapter 8} introduces advanced temporal graph neural networks for network security, developing two complementary frameworks that capture attack propagation patterns across dynamic graph structures. The Continuous-Time Temporal Graph Neural Network (CT-TGNN) framework integrates graph structure evolution with continuous node dynamics through neural graph ordinary differential equations, enabling detection of sophisticated temporal attack patterns including lateral movement and privilege escalation across time-varying service dependencies. The architecture achieves 96.4\% accuracy on microservices security graphs with six diverse datasets including realistic applications with 41 to 100 services. The Triple-Embedding Temporal Graph Neural Network (TripleE-TGNN) framework addresses multi-granularity microservices security by jointly learning hierarchical representations at service-level (aggregate behaviors through GAT and GRU), trace-level (distributed request flows via time-aware GCN), and node-level (pod dynamics through temporal GCN and LSTM) granularities. A heterogeneous temporal graph neural network integrates these complementary embeddings with adaptive granularity fusion, achieving 96.8\% accuracy on Train-Ticket with 41 services and demonstrating that each granularity contributes unique discriminative power with service-level excelling at DDoS detection (98.2\%), trace-level at privilege escalation (97.4\%), and node-level at container breakout (94.7\%). Comprehensive mathematical formulations describe continuous-time graph dynamics, multi-scale temporal attention, heterogeneous message passing, and adaptive cross-granularity fusion mechanisms.

\textbf{Chapter 9} provides comprehensive unified experimental evaluation and comparative analysis across all nine research contributions developed in this dissertation. The chapter presents results on 25 diverse datasets spanning traditional network intrusion detection benchmarks (CIC-IDS2018, UNSW-NB15, NSL-KDD), microservices architectures (Train-Ticket with 41 services, Sock-Shop, Online-Boutique, DeathStarBench suite), federated API security (AWS, Azure, Google Cloud APIs), post-quantum encrypted traffic (CESNET-TLS-22 with 2.1M hybrid connections, QUIC-PQC, IoT-PQC), container orchestration (Kubernetes flows), IoT/IIoT networks, and enterprise security operations. Detailed performance analysis presents accuracy, precision, recall, F1-score, and false positive rates across all methods, with statistical significance testing validating improvements. Attack-specific breakdowns quantify performance across 12 threat categories including DDoS, privilege escalation, API abuse, container breakout, lateral movement, resource exhaustion, and quantum-enhanced evasion. Extensive ablation studies systematically evaluate each architectural component across all frameworks, quantifying contributions of temporal adaptive normalization, continuous-time graph dynamics, Byzantine-robust aggregation, parameter-efficient federated learning, quantum-classical fusion, and multi-granularity embedding. Computational efficiency analysis reports inference latency, training time, memory consumption, and communication overhead. Cross-domain validation on speech event detection and healthcare monitoring demonstrates broad applicability. Robustness evaluation assesses performance under concept drift, adversarial attacks, Byzantine participants, and quantum threats. Comparison tables present side-by-side results against 18 baseline methods including TGAT, DySAT, CONTINUUM, standard GNN architectures, classical federated learning, and existing microservices security approaches.

\textbf{Chapter 10} concludes the dissertation with a comprehensive summary of contributions across nine complementary research directions, discussion of implications for network security practice spanning deployment considerations for continuous-time models, federated learning infrastructure requirements, post-quantum transition strategies, and microservices observability integration, acknowledgment of limitations including quantum hardware availability, dataset coverage gaps, and computational overhead for resource-constrained environments, and identification of promising directions for future research. Novel future directions suitable for postdoctoral investigation include extending continuous-time graph neural networks to hypergraphs for modeling complex multi-way interactions, developing federated learning for foundation models beyond language models including vision transformers and multimodal architectures, establishing adversarial robustness for hybrid quantum-classical systems against noise-resilient quantum attacks, integrating zero-knowledge proofs for verifiable federated learning, advancing multi-granularity modeling to additional hierarchies including data center, rack, and hardware levels, and exploring neuro-symbolic approaches combining neural temporal graph learning with formal verification for certified security guarantees.

\textbf{Appendices} provide supplementary material including detailed mathematical proofs for convergence theorems, privacy analyses, and robustness guarantees, algorithmic pseudocode for all major contributions with complexity analysis, implementation details including hyperparameter configurations and training procedures, comprehensive dataset descriptions with data collection methodologies and preprocessing pipelines, additional experimental results including extended ablation studies and sensitivity analyses, and TikZ architectural diagrams illustrating neural ordinary differential equation integrators, optimal transport flow, federated aggregation topology, quantum circuit layouts, and multi-granularity graph structures.

\section{Summary}

This chapter has introduced the nine fundamental research problems addressed in this dissertation spanning continuous-time temporal modeling, privacy-preserving optimal transport, encrypted traffic analysis, federated learning, continuous-time graph neural networks, large language model federation for API security, post-quantum adversarial robustness, and multi-granularity microservices security. The research objectives establish a comprehensive framework for advancing network intrusion detection through modern machine learning techniques including neural ordinary differential equations, optimal transport theory, quantum machine learning, graph neural networks, and large language models. Key contributions encompass theoretical developments with convergence guarantees and privacy proofs, algorithmic innovations achieving state-of-the-art performance across diverse security domains, and empirical validation on 25 datasets demonstrating practical deployment viability. The organizational roadmap provides a structured presentation of technical developments, mathematical foundations, comprehensive experimental analysis, and thorough discussion of implications for cybersecurity practice. The following chapters present detailed formulations, architectures, algorithms, and evaluations for each research contribution.
