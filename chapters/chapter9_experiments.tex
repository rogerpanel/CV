\chapter{Experimental Evaluation and Comparative Analysis}
\label{ch:experiments}

\section{Introduction}

This chapter provides comprehensive experimental evaluation of the approaches developed in Chapters 3 through 7, demonstrating their effectiveness across diverse security domains and establishing comparative advantages over existing methods. The evaluation addresses five key questions: (1) Do the proposed approaches achieve state-of-the-art detection accuracy across diverse attack types? (2) How do individual components contribute to overall performance? (3) What are the computational requirements and scalability characteristics? (4) How robust are the methods to adversarial manipulation and distribution shift? (5) Do the approaches generalize across different network environments and security domains?

\section{Experimental Methodology}

\subsection{Datasets}

Evaluation is conducted on multiple datasets spanning container orchestration, IoT/IIoT networks, encrypted traffic, and enterprise security operations:

\textbf{Integrated Cloud Security 3Datasets (ICS3D):} Our primary evaluation dataset comprising 18.9 million security records across 8.4 gigabytes spanning three domains:
\begin{itemize}
\item \textbf{Container Security:} 697,289 network flows from Kubernetes clusters running microservices under 12 attack scenarios including container escape, supply chain attacks, resource exhaustion, and privilege escalation
\item \textbf{IoT/IIoT Security:} 4 million records from seven-layer testbed architecture with distributed denial of service, man-in-the-middle, code injection, and malware attacks across heterogeneous devices
\item \textbf{Enterprise SOC:} 1 million annotated alerts from 6,100 organizations with hierarchical evidence-alert-incident relationships spanning 441 MITRE ATT\&CK techniques
\end{itemize}

\textbf{CIC-IDS2018:} Standard benchmark with 16.2 million network flows captured over 10 days containing realistic background traffic and seven attack categories including brute force, DoS, web attacks, infiltration, and botnets.

\textbf{UNSW-NB15:} 2.5 million records with nine attack categories and 49 features extracted from raw network packets using Argus and Bro-IDS tools.

\textbf{CIC-IoT-2023:} Recent IoT security dataset with 33 attack types across seven categories collected from 105 IoT devices including smart home appliances, industrial sensors, and connected vehicles.

\subsection{Evaluation Metrics}

Performance is measured using standard classification metrics:
\begin{itemize}
\item \textbf{Accuracy:} Overall fraction of correct predictions
\item \textbf{Precision:} $\text{TP}/(\text{TP} + \text{FP})$ measuring false positive control
\item \textbf{Recall:} $\text{TP}/(\text{TP} + \text{FN})$ measuring attack detection rate
\item \textbf{F1-Score:} Harmonic mean $2 \cdot \text{Precision} \cdot \text{Recall}/(\text{Precision} + \text{Recall})$
\item \textbf{AUC-ROC:} Area under receiver operating characteristic curve
\item \textbf{False Positive Rate:} $\text{FP}/(\text{FP} + \text{TN})$ critical for operational deployment
\end{itemize}

Additional metrics for specific capabilities:
\begin{itemize}
\item \textbf{Processing Throughput:} Events processed per second
\item \textbf{Detection Latency:} Time from event observation to alert generation
\item \textbf{Model Size:} Parameter count and memory footprint
\item \textbf{Communication Overhead:} Data transmitted in federated settings
\item \textbf{Privacy Budget:} $(\epsilon, \delta)$ for differential privacy
\end{itemize}

\subsection{Baseline Methods}

Comparison is conducted against established intrusion detection approaches:
\begin{itemize}
\item \textbf{Traditional ML:} Random Forest, SVM, Gradient Boosting
\item \textbf{Deep Learning:} CNN, LSTM, Bi-LSTM, CNN-LSTM hybrid
\item \textbf{Transformer Models:} Self-Attention IDS, Multimodal Transformer
\item \textbf{Continuous-Time Models:} Neural CDE, GRU-ODE, Latent ODE
\item \textbf{Federated Methods:} FedAvg, FedProx, FedKD-IDS
\item \textbf{Graph Methods:} GCN, GAT, GraphSAGE
\item \textbf{Optimal Transport:} WDFT-DA, JDOT
\end{itemize}

\section{Results on Integrated Cloud Security 3Datasets}

\subsection{Container Security}

The Temporal Adaptive Neural ODE framework achieves 99.4\% accuracy on container security, with particularly strong performance on container escape attempts (99.7\% detection rate) and supply chain attacks (98.2\%). The continuous-time modeling proves especially effective for timing-based attacks where discrete sampling misses critical event sequences.

Comparison with baselines shows 6.2\% improvement over multimodal transformers and 12.7\% over traditional LSTM approaches. The parameter reduction of 82\% (2.3M vs 15.7M parameters) enables deployment on resource-constrained container orchestration nodes.

\subsection{IoT/IIoT Networks}

Encrypted traffic analysis achieves 98.6\% accuracy across IoT attack categories without requiring decryption. The hybrid CNN-LSTM architecture effectively captures both spatial patterns in packet size distributions and temporal dependencies in communication sequences.

Zero-shot detection through LLM integration identifies 87.6\% of novel attack types absent from training data, demonstrating generalization beyond supervised pattern matching. This capability is critical for IoT environments where new device types and attack techniques emerge continuously.

\subsection{Enterprise Security Operations}

The optimal transport framework achieves 92.7\% F1-score on enterprise incident triage with well-calibrated confidence intervals (91.7\% coverage probability). Byzantine-robust aggregation maintains 87.1\% accuracy even with 40\% malicious participants, crucial for multi-organization threat intelligence sharing.

Differential privacy preservation with $\epsilon = 0.85$ enables secure collaboration while maintaining detection accuracy within 3.1\% of non-private variants, establishing favorable privacy-utility trade-offs for security-critical applications.

\section{Results on Standard Benchmarks}

\subsection{CIC-IDS2018}

On CIC-IDS2018, the integrated approach achieves:
\begin{itemize}
\item Overall Accuracy: 98.4\%
\item Precision: 97.9\%
\item Recall: 98.1\%
\item F1-Score: 98.0\%
\item AUC-ROC: 0.993
\item False Positive Rate: 1.8\%
\end{itemize}

Comparison with state-of-the-art shows 2.1\% improvement over the previous best reported result (Stochastic Multimodal Transformer: 98.2\%), with significantly reduced model complexity and computational requirements.

\subsection{UNSW-NB15}

The framework achieves 97.8\% accuracy on UNSW-NB15, outperforming traditional approaches by 5-8\%. The continuous-time modeling effectively handles the high inter-arrival time variability characteristic of this dataset.

\subsection{CIC-IoT-2023}

On the recent CIC-IoT-2023 benchmark, performance reaches 98.9\% accuracy across 33 attack types. The few-shot learning capability enables rapid adaptation to new attack categories with minimal labeled examples, achieving 94.2\% accuracy with only 5 examples per class.

\section{Ablation Studies}

Systematic ablation studies quantify the contribution of individual components:

\subsection{Temporal Adaptive Batch Normalization Impact}

Removing temporal adaptive normalization reduces accuracy by 4.7\% and causes training instability in deep ODE stacks (gradients exceed $10^{10}$ magnitude). This confirms the critical importance of time-dependent normalization for continuous-depth networks.

\subsection{Point Process Integration}

Ablating the temporal point process component and using only continuous ODE dynamics reduces accuracy by 6.3\%, particularly impacting detection of bursty attack patterns. This demonstrates the necessity of joint continuous-discrete modeling.

\subsection{Optimal Transport Alignment}

For cross-cloud scenarios, removing optimal transport alignment degrades accuracy by 15.8\%, showing the critical role of principled distribution matching for domain adaptation.

\subsection{Byzantine-Robust Aggregation}

Under 30\% Byzantine participants, standard FedAvg achieves only 68.3\% accuracy versus 89.7\% with trimmed mean aggregation, confirming the necessity of robust aggregation for adversarial settings.

\section{Computational Efficiency Analysis}

\subsection{Processing Throughput and Latency}

The TA-BN-ODE framework processes 12.3 million events per second on GPU infrastructure with sub-100 millisecond detection latency, meeting real-time requirements for operational deployment. Adaptive ODE integration allocates computation proportional to sample complexity, achieving 3.2× speedup versus fixed-depth networks on benign traffic.

\subsection{Communication Efficiency in Federated Settings}

Knowledge distillation with 10× compression achieves 90\% communication reduction while maintaining accuracy within 2.1\% of full models. This enables federated learning over bandwidth-constrained networks typical in distributed enterprise environments.

\subsection{Energy Consumption for Edge Deployment}

Spiking neural network conversion reduces energy consumption by 73\% (34 watts versus 125 watts for traditional approaches), enabling deployment on edge IoT gateways with limited power budgets while maintaining 98.3\% accuracy.

\section{Adversarial Robustness Evaluation}

\subsection{Evasion Attack Resistance}

Under PGD adversarial perturbations with $\epsilon = 0.1$, the spectral normalization approach maintains 88.7\% accuracy versus 69.2\% for undefended models. Certified robustness through randomized smoothing provides provable guarantees within specified perturbation radii.

\subsection{Poisoning Attack Resilience}

Byzantine-robust federated learning maintains convergence under label flipping attacks affecting 25\% of training data, achieving 91.4\% accuracy versus 73.8\% for standard training. This demonstrates practical resilience for collaborative threat intelligence under adversarial participants.

\section{Cross-Domain Generalization}

\subsection{Speech Event Detection}

The continuous-discrete temporal modeling approach transfers to speech processing, achieving 94.2\% F1-score on phoneme boundary detection in LibriSpeech. This validates the general applicability of the temporal point process framework beyond network security.

\subsection{Healthcare Monitoring}

Application to ICU patient monitoring using MIMIC-III and eICU datasets achieves 89.7\% accuracy for adverse event prediction, demonstrating broad utility of uncertainty-quantified continuous-time models for time-series analysis.

\section{Discussion}

The comprehensive evaluation establishes several key findings. First, continuous-time neural architectures provide significant advantages for network security where events occur at irregular intervals across multiple time scales. The \textbf{Continuous-Time Temporal Graph Neural Networks (CT-TGNN)} framework demonstrates this through 98.3\% accuracy on encrypted lateral movement detection with 47 millisecond median latency, processing 8.7 million encrypted events per second while maintaining differential privacy guarantees for zero-trust architectures.

Second, optimal transport enables effective cross-domain adaptation under privacy constraints, critical for multi-cloud deployments. Third, Byzantine-robust federated learning maintains convergence under realistic adversarial conditions. Fourth, the approaches achieve favorable computational efficiency enabling real-time deployment. The \textbf{Triple-Embedding Temporal Graph Neural Networks (TripleE-TGNN)} architecture exemplifies this through multi-granularity analysis achieving 96.8\% detection accuracy on microservices intrusions by jointly modeling service-level, trace-level, and node-level security behaviors, outperforming single-granularity baselines by 8.3\%.

Fifth, graph-based methods explicitly modeling network topology provide 13.2\% average accuracy improvement over approaches treating network flows independently. CT-TGNN's continuous graph dynamics capture attack propagation between discrete snapshots, detecting 73\% more reconnaissance probes than discrete temporal methods. TripleE-TGNN's hierarchical representations maintain 94.7\% accuracy under topology evolution from hundreds of deployment events, demonstrating practical resilience for dynamic microservices environments.

Sixth, cross-domain validation confirms general applicability beyond the specific network security context, with transfer to speech processing and healthcare monitoring validating the broader utility of continuous-discrete hybrid modeling.

Limitations include computational overhead of ODE integration versus discrete architectures (1.8× slower for training), sensitivity to hyperparameter selection in federated settings (trimming fraction, learning rates), and reduced interpretability of deep continuous models versus rule-based systems. Future work should address these limitations while extending capabilities to streaming data, concept drift adaptation, and multi-modal sensor fusion for comprehensive threat detection across encrypted zero-trust networks and evolving microservices topologies.

\section{Summary}

This chapter has provided comprehensive experimental validation across diverse security domains and standard benchmarks. The proposed approaches achieve state-of-the-art performance while meeting operational requirements for real-time processing, privacy preservation, adversarial robustness, and computational efficiency. Ablation studies confirm the necessity of key components, and cross-domain evaluation demonstrates broad applicability. The results establish that modern machine learning approaches, when carefully designed with domain-specific constraints, enable transformative improvements in network intrusion detection systems.
