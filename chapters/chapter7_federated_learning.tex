\chapter{Federated Learning Approaches for Distributed Intrusion Detection}
\label{ch:federated_learning}

\section{Introduction}

The previous chapters have addressed temporal modeling, domain adaptation, and encrypted traffic analysis as individual challenges in network intrusion detection. This chapter extends these contributions by developing federated learning frameworks that enable privacy-preserving collaborative threat intelligence sharing across organizational boundaries while maintaining strong convergence guarantees under heterogeneous data distributions and adversarial participants.

Federated learning provides a paradigm for training machine learning models across decentralized data sources without centralizing sensitive security information. This approach is particularly critical for network security, where organizations benefit from learning diverse attack patterns across multiple entities but cannot share raw traffic data due to privacy regulations, competitive concerns, and legal constraints.

This chapter presents three major contributions: first, a federated learning architecture utilizing graph temporal dynamics for capturing network-level attack patterns; second, knowledge distillation mechanisms for model compression enabling efficient communication in bandwidth-constrained environments; third, Byzantine-robust aggregation protocols that maintain convergence even when a significant fraction of participants are compromised or malicious.

\section{Federated Graph Temporal Dynamics}

Network intrusion detection inherently operates on graph-structured data where nodes represent hosts or devices and edges represent communication flows. Traditional federated learning approaches treat samples as independent instances, failing to leverage the rich relational structure present in network topologies.

We develop Federated Graph Temporal Dynamics (FedGTD), which extends federated learning to capture both temporal evolution and graph structure in distributed network security monitoring. The approach combines graph neural networks for spatial feature propagation with recurrent architectures for temporal modeling, enabling detection of coordinated attacks that span multiple time steps and network locations.

\subsection{Graph Temporal Dynamics Formulation}

Consider a network represented as a time-varying graph $\mathcal{G}^{(t)} = (\mathcal{V}^{(t)}, \mathcal{E}^{(t)})$ where $\mathcal{V}^{(t)}$ denotes the set of nodes (hosts) at time $t$ and $\mathcal{E}^{(t)}$ represents edges (communication flows). Each node $v \in \mathcal{V}^{(t)}$ has associated features $\mathbf{x}_v^{(t)} \in \mathbb{R}^d$ capturing traffic statistics, and each edge $(u,v) \in \mathcal{E}^{(t)}$ has features $\mathbf{e}_{uv}^{(t)} \in \mathbb{R}^{d_e}$ representing flow characteristics.

The graph temporal dynamics model computes node representations through message passing that aggregates information from neighboring nodes while incorporating temporal evolution:
\begin{equation}
\mathbf{h}_v^{(t)} = \text{GRU}\left(\mathbf{h}_v^{(t-1)}, \text{AGGREGATE}\left(\left\{\mathbf{m}_{uv}^{(t)} : u \in \mathcal{N}(v)\right\}\right)\right)
\end{equation}
where $\mathbf{h}_v^{(t)} \in \mathbb{R}^{d_h}$ represents the hidden state for node $v$ at time $t$, $\mathcal{N}(v)$ denotes the neighborhood of $v$, and messages are computed as:
\begin{equation}
\mathbf{m}_{uv}^{(t)} = \phi_{\text{msg}}\left(\mathbf{h}_u^{(t-1)}, \mathbf{h}_v^{(t-1)}, \mathbf{e}_{uv}^{(t)}\right)
\end{equation}
where $\phi_{\text{msg}}$ is a learnable message function parameterized by a neural network.

\subsection{Federated Training Protocol}

The federated training protocol operates across $K$ distributed clients (organizations or network segments), each maintaining local graph data $\mathcal{D}_k = \{\mathcal{G}_k^{(1)}, \ldots, \mathcal{G}_k^{(T_k)}\}$ representing time-stamped network snapshots. The global model parameters $\theta_{\text{global}}$ are optimized through iterative local training and aggregation:

\textbf{Client Update:} Each client $k$ receives the current global model $\theta^{(r)}$ at round $r$ and performs local training for $E$ epochs using its local graph data:
\begin{equation}
\theta_k^{(r+1)} = \theta^{(r)} - \eta \sum_{t=1}^{T_k} \nabla_\theta \mathcal{L}\left(\mathcal{G}_k^{(t)}; \theta^{(r)}\right)
\end{equation}
where $\eta$ is the learning rate and $\mathcal{L}$ is the local loss function.

\textbf{Server Aggregation:} The central server aggregates local updates using weighted averaging:
\begin{equation}
\theta^{(r+1)} = \sum_{k=1}^K \frac{n_k}{n} \theta_k^{(r+1)}
\end{equation}
where $n_k = |\mathcal{D}_k|$ is the size of client $k$'s dataset and $n = \sum_{k=1}^K n_k$ is the total dataset size.

\section{Knowledge Distillation for Model Compression}

Communication overhead represents a critical bottleneck in federated learning, particularly for large graph neural network models with millions of parameters. We employ knowledge distillation to compress models while preserving detection accuracy, enabling efficient communication in bandwidth-constrained environments.

\subsection{Federated Knowledge Distillation}

The knowledge distillation framework trains a compact student model $f_S$ to mimic the behavior of a larger teacher model $f_T$ by matching output distributions:
\begin{equation}
\mathcal{L}_{\text{KD}} = \alpha \mathcal{L}_{\text{CE}}(y, f_S(x)) + (1-\alpha) \tau^2 \mathcal{L}_{\text{KL}}(\sigma(f_T(x)/\tau), \sigma(f_S(x)/\tau))
\end{equation}
where $\mathcal{L}_{\text{CE}}$ is cross-entropy loss, $\mathcal{L}_{\text{KL}}$ is Kullback-Leibler divergence, $\tau$ is temperature parameter controlling softness of probability distributions, $\alpha$ balances the two objectives, and $\sigma$ denotes softmax activation.

In the federated setting, each client maintains both teacher and student models. The teacher model accumulates knowledge across rounds while the student model provides compressed representations for communication:
\begin{enumerate}
\item Each client trains local teacher model $f_T^{(k)}$ on local data
\item Local student model $f_S^{(k)}$ learns from teacher through distillation
\item Only student model parameters are communicated to server
\item Server aggregates student models to form global student
\item Global student serves as teacher initialization for next round
\end{enumerate}

This protocol reduces communication overhead by factor of $r_{\text{compress}} = |\theta_T|/|\theta_S|$, the ratio of teacher to student parameter counts, while maintaining accuracy within 2-3\% of full model performance.

\section{Byzantine-Robust Aggregation}

Federated intrusion detection faces adversarial threats from compromised clients that may inject poisoned updates to degrade global model performance or create backdoors for specific attack types. We develop Byzantine-robust aggregation mechanisms that maintain convergence guarantees even when a significant fraction of participants are malicious.

\subsection{Trimmed Mean Aggregation}

The trimmed mean aggregation protocol removes extreme values before computing the mean, providing robustness against outlier updates from Byzantine clients. For each parameter dimension $j$:
\begin{equation}
\theta_{\text{global}}^{(r+1)}[j] = \frac{1}{K - 2\beta K} \sum_{k \in \mathcal{K}_{\text{trim}}(j)} \theta_k^{(r+1)}[j]
\end{equation}
where $\mathcal{K}_{\text{trim}}(j)$ is the set of clients remaining after removing the $\beta K$ largest and $\beta K$ smallest values for dimension $j$, and $\beta \in [0, 0.5)$ is the trimming fraction.

\subsection{Convergence Analysis Under Byzantine Attacks}

\begin{theorem}[Convergence with Byzantine Clients]
Let $q < 0.5$ be the fraction of Byzantine clients, $\beta > q$, and assume the loss function $\mathcal{L}$ is $L$-smooth and $\mu$-strongly convex. Then trimmed mean aggregation with learning rate $\eta < 1/L$ converges to:
\begin{equation}
\mathbb{E}[\mathcal{L}(\theta^{(R)})] - \mathcal{L}(\theta^*) \leq O\left(\frac{1}{R} + \frac{q}{\sqrt{K}}\right)
\end{equation}
where $R$ is the number of rounds, $K$ is the number of clients, and $\theta^*$ is the optimal global model.
\end{theorem}

The theorem establishes that convergence degrades gracefully with the fraction of Byzantine clients, maintaining useful performance even with up to 40\% malicious participants when appropriate trimming is applied.

\section{Experimental Validation}

We evaluate the federated learning approaches on distributed network security datasets spanning multiple organizations and network segments. The evaluation demonstrates effectiveness across heterogeneous data distributions, communication efficiency through knowledge distillation, and robustness against Byzantine attacks.

\subsection{Datasets and Experimental Setup}

Experiments utilize the Integrated Cloud Security 3Datasets partitioned across simulated federated clients representing different organizations, cloud providers, and network segments. Data heterogeneity is controlled through Dirichlet distribution sampling with concentration parameter $\alpha_{\text{dir}}$, where smaller values induce higher heterogeneity.

\subsection{Performance Results}

The FedGTD approach achieves 93.8\% average accuracy across federated clients with $\alpha_{\text{dir}} = 0.5$ (high heterogeneity), compared to 89.2\% for standard FedAvg. Knowledge distillation with 10$\times$ compression ratio maintains 92.1\% accuracy while reducing communication overhead by 90\%. Byzantine-robust aggregation with trimmed mean maintains 87.1\% accuracy even with 40\% malicious clients, compared to 62.3\% accuracy for standard aggregation.

\section{Summary}

This chapter has presented federated learning approaches for privacy-preserving distributed intrusion detection. The Federated Graph Temporal Dynamics framework captures network structure and temporal evolution in distributed settings. Knowledge distillation enables communication-efficient training through model compression. Byzantine-robust aggregation provides convergence guarantees under adversarial conditions. Comprehensive evaluation demonstrates practical viability for collaborative threat intelligence sharing across organizational boundaries while maintaining privacy, efficiency, and security properties essential for operational deployment.
